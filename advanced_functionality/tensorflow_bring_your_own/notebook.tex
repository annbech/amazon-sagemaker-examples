
% Default to the notebook output style

    


% Inherit from the specified cell style.




    
\documentclass[11pt]{article}

    
    
    \usepackage[T1]{fontenc}
    % Nicer default font (+ math font) than Computer Modern for most use cases
    \usepackage{mathpazo}

    % Basic figure setup, for now with no caption control since it's done
    % automatically by Pandoc (which extracts ![](path) syntax from Markdown).
    \usepackage{graphicx}
    % We will generate all images so they have a width \maxwidth. This means
    % that they will get their normal width if they fit onto the page, but
    % are scaled down if they would overflow the margins.
    \makeatletter
    \def\maxwidth{\ifdim\Gin@nat@width>\linewidth\linewidth
    \else\Gin@nat@width\fi}
    \makeatother
    \let\Oldincludegraphics\includegraphics
    % Set max figure width to be 80% of text width, for now hardcoded.
    \renewcommand{\includegraphics}[1]{\Oldincludegraphics[width=.8\maxwidth]{#1}}
    % Ensure that by default, figures have no caption (until we provide a
    % proper Figure object with a Caption API and a way to capture that
    % in the conversion process - todo).
    \usepackage{caption}
    \DeclareCaptionLabelFormat{nolabel}{}
    \captionsetup{labelformat=nolabel}

    \usepackage{adjustbox} % Used to constrain images to a maximum size 
    \usepackage{xcolor} % Allow colors to be defined
    \usepackage{enumerate} % Needed for markdown enumerations to work
    \usepackage{geometry} % Used to adjust the document margins
    \usepackage{amsmath} % Equations
    \usepackage{amssymb} % Equations
    \usepackage{textcomp} % defines textquotesingle
    % Hack from http://tex.stackexchange.com/a/47451/13684:
    \AtBeginDocument{%
        \def\PYZsq{\textquotesingle}% Upright quotes in Pygmentized code
    }
    \usepackage{upquote} % Upright quotes for verbatim code
    \usepackage{eurosym} % defines \euro
    \usepackage[mathletters]{ucs} % Extended unicode (utf-8) support
    \usepackage[utf8x]{inputenc} % Allow utf-8 characters in the tex document
    \usepackage{fancyvrb} % verbatim replacement that allows latex
    \usepackage{grffile} % extends the file name processing of package graphics 
                         % to support a larger range 
    % The hyperref package gives us a pdf with properly built
    % internal navigation ('pdf bookmarks' for the table of contents,
    % internal cross-reference links, web links for URLs, etc.)
    \usepackage{hyperref}
    \usepackage{longtable} % longtable support required by pandoc >1.10
    \usepackage{booktabs}  % table support for pandoc > 1.12.2
    \usepackage[inline]{enumitem} % IRkernel/repr support (it uses the enumerate* environment)
    \usepackage[normalem]{ulem} % ulem is needed to support strikethroughs (\sout)
                                % normalem makes italics be italics, not underlines
    

    
    
    % Colors for the hyperref package
    \definecolor{urlcolor}{rgb}{0,.145,.698}
    \definecolor{linkcolor}{rgb}{.71,0.21,0.01}
    \definecolor{citecolor}{rgb}{.12,.54,.11}

    % ANSI colors
    \definecolor{ansi-black}{HTML}{3E424D}
    \definecolor{ansi-black-intense}{HTML}{282C36}
    \definecolor{ansi-red}{HTML}{E75C58}
    \definecolor{ansi-red-intense}{HTML}{B22B31}
    \definecolor{ansi-green}{HTML}{00A250}
    \definecolor{ansi-green-intense}{HTML}{007427}
    \definecolor{ansi-yellow}{HTML}{DDB62B}
    \definecolor{ansi-yellow-intense}{HTML}{B27D12}
    \definecolor{ansi-blue}{HTML}{208FFB}
    \definecolor{ansi-blue-intense}{HTML}{0065CA}
    \definecolor{ansi-magenta}{HTML}{D160C4}
    \definecolor{ansi-magenta-intense}{HTML}{A03196}
    \definecolor{ansi-cyan}{HTML}{60C6C8}
    \definecolor{ansi-cyan-intense}{HTML}{258F8F}
    \definecolor{ansi-white}{HTML}{C5C1B4}
    \definecolor{ansi-white-intense}{HTML}{A1A6B2}

    % commands and environments needed by pandoc snippets
    % extracted from the output of `pandoc -s`
    \providecommand{\tightlist}{%
      \setlength{\itemsep}{0pt}\setlength{\parskip}{0pt}}
    \DefineVerbatimEnvironment{Highlighting}{Verbatim}{commandchars=\\\{\}}
    % Add ',fontsize=\small' for more characters per line
    \newenvironment{Shaded}{}{}
    \newcommand{\KeywordTok}[1]{\textcolor[rgb]{0.00,0.44,0.13}{\textbf{{#1}}}}
    \newcommand{\DataTypeTok}[1]{\textcolor[rgb]{0.56,0.13,0.00}{{#1}}}
    \newcommand{\DecValTok}[1]{\textcolor[rgb]{0.25,0.63,0.44}{{#1}}}
    \newcommand{\BaseNTok}[1]{\textcolor[rgb]{0.25,0.63,0.44}{{#1}}}
    \newcommand{\FloatTok}[1]{\textcolor[rgb]{0.25,0.63,0.44}{{#1}}}
    \newcommand{\CharTok}[1]{\textcolor[rgb]{0.25,0.44,0.63}{{#1}}}
    \newcommand{\StringTok}[1]{\textcolor[rgb]{0.25,0.44,0.63}{{#1}}}
    \newcommand{\CommentTok}[1]{\textcolor[rgb]{0.38,0.63,0.69}{\textit{{#1}}}}
    \newcommand{\OtherTok}[1]{\textcolor[rgb]{0.00,0.44,0.13}{{#1}}}
    \newcommand{\AlertTok}[1]{\textcolor[rgb]{1.00,0.00,0.00}{\textbf{{#1}}}}
    \newcommand{\FunctionTok}[1]{\textcolor[rgb]{0.02,0.16,0.49}{{#1}}}
    \newcommand{\RegionMarkerTok}[1]{{#1}}
    \newcommand{\ErrorTok}[1]{\textcolor[rgb]{1.00,0.00,0.00}{\textbf{{#1}}}}
    \newcommand{\NormalTok}[1]{{#1}}
    
    % Additional commands for more recent versions of Pandoc
    \newcommand{\ConstantTok}[1]{\textcolor[rgb]{0.53,0.00,0.00}{{#1}}}
    \newcommand{\SpecialCharTok}[1]{\textcolor[rgb]{0.25,0.44,0.63}{{#1}}}
    \newcommand{\VerbatimStringTok}[1]{\textcolor[rgb]{0.25,0.44,0.63}{{#1}}}
    \newcommand{\SpecialStringTok}[1]{\textcolor[rgb]{0.73,0.40,0.53}{{#1}}}
    \newcommand{\ImportTok}[1]{{#1}}
    \newcommand{\DocumentationTok}[1]{\textcolor[rgb]{0.73,0.13,0.13}{\textit{{#1}}}}
    \newcommand{\AnnotationTok}[1]{\textcolor[rgb]{0.38,0.63,0.69}{\textbf{\textit{{#1}}}}}
    \newcommand{\CommentVarTok}[1]{\textcolor[rgb]{0.38,0.63,0.69}{\textbf{\textit{{#1}}}}}
    \newcommand{\VariableTok}[1]{\textcolor[rgb]{0.10,0.09,0.49}{{#1}}}
    \newcommand{\ControlFlowTok}[1]{\textcolor[rgb]{0.00,0.44,0.13}{\textbf{{#1}}}}
    \newcommand{\OperatorTok}[1]{\textcolor[rgb]{0.40,0.40,0.40}{{#1}}}
    \newcommand{\BuiltInTok}[1]{{#1}}
    \newcommand{\ExtensionTok}[1]{{#1}}
    \newcommand{\PreprocessorTok}[1]{\textcolor[rgb]{0.74,0.48,0.00}{{#1}}}
    \newcommand{\AttributeTok}[1]{\textcolor[rgb]{0.49,0.56,0.16}{{#1}}}
    \newcommand{\InformationTok}[1]{\textcolor[rgb]{0.38,0.63,0.69}{\textbf{\textit{{#1}}}}}
    \newcommand{\WarningTok}[1]{\textcolor[rgb]{0.38,0.63,0.69}{\textbf{\textit{{#1}}}}}
    
    
    % Define a nice break command that doesn't care if a line doesn't already
    % exist.
    \def\br{\hspace*{\fill} \\* }
    % Math Jax compatability definitions
    \def\gt{>}
    \def\lt{<}
    % Document parameters
    \title{tensorflow\_bring\_your\_own}
    
    
    

    % Pygments definitions
    
\makeatletter
\def\PY@reset{\let\PY@it=\relax \let\PY@bf=\relax%
    \let\PY@ul=\relax \let\PY@tc=\relax%
    \let\PY@bc=\relax \let\PY@ff=\relax}
\def\PY@tok#1{\csname PY@tok@#1\endcsname}
\def\PY@toks#1+{\ifx\relax#1\empty\else%
    \PY@tok{#1}\expandafter\PY@toks\fi}
\def\PY@do#1{\PY@bc{\PY@tc{\PY@ul{%
    \PY@it{\PY@bf{\PY@ff{#1}}}}}}}
\def\PY#1#2{\PY@reset\PY@toks#1+\relax+\PY@do{#2}}

\expandafter\def\csname PY@tok@w\endcsname{\def\PY@tc##1{\textcolor[rgb]{0.73,0.73,0.73}{##1}}}
\expandafter\def\csname PY@tok@c\endcsname{\let\PY@it=\textit\def\PY@tc##1{\textcolor[rgb]{0.25,0.50,0.50}{##1}}}
\expandafter\def\csname PY@tok@cp\endcsname{\def\PY@tc##1{\textcolor[rgb]{0.74,0.48,0.00}{##1}}}
\expandafter\def\csname PY@tok@k\endcsname{\let\PY@bf=\textbf\def\PY@tc##1{\textcolor[rgb]{0.00,0.50,0.00}{##1}}}
\expandafter\def\csname PY@tok@kp\endcsname{\def\PY@tc##1{\textcolor[rgb]{0.00,0.50,0.00}{##1}}}
\expandafter\def\csname PY@tok@kt\endcsname{\def\PY@tc##1{\textcolor[rgb]{0.69,0.00,0.25}{##1}}}
\expandafter\def\csname PY@tok@o\endcsname{\def\PY@tc##1{\textcolor[rgb]{0.40,0.40,0.40}{##1}}}
\expandafter\def\csname PY@tok@ow\endcsname{\let\PY@bf=\textbf\def\PY@tc##1{\textcolor[rgb]{0.67,0.13,1.00}{##1}}}
\expandafter\def\csname PY@tok@nb\endcsname{\def\PY@tc##1{\textcolor[rgb]{0.00,0.50,0.00}{##1}}}
\expandafter\def\csname PY@tok@nf\endcsname{\def\PY@tc##1{\textcolor[rgb]{0.00,0.00,1.00}{##1}}}
\expandafter\def\csname PY@tok@nc\endcsname{\let\PY@bf=\textbf\def\PY@tc##1{\textcolor[rgb]{0.00,0.00,1.00}{##1}}}
\expandafter\def\csname PY@tok@nn\endcsname{\let\PY@bf=\textbf\def\PY@tc##1{\textcolor[rgb]{0.00,0.00,1.00}{##1}}}
\expandafter\def\csname PY@tok@ne\endcsname{\let\PY@bf=\textbf\def\PY@tc##1{\textcolor[rgb]{0.82,0.25,0.23}{##1}}}
\expandafter\def\csname PY@tok@nv\endcsname{\def\PY@tc##1{\textcolor[rgb]{0.10,0.09,0.49}{##1}}}
\expandafter\def\csname PY@tok@no\endcsname{\def\PY@tc##1{\textcolor[rgb]{0.53,0.00,0.00}{##1}}}
\expandafter\def\csname PY@tok@nl\endcsname{\def\PY@tc##1{\textcolor[rgb]{0.63,0.63,0.00}{##1}}}
\expandafter\def\csname PY@tok@ni\endcsname{\let\PY@bf=\textbf\def\PY@tc##1{\textcolor[rgb]{0.60,0.60,0.60}{##1}}}
\expandafter\def\csname PY@tok@na\endcsname{\def\PY@tc##1{\textcolor[rgb]{0.49,0.56,0.16}{##1}}}
\expandafter\def\csname PY@tok@nt\endcsname{\let\PY@bf=\textbf\def\PY@tc##1{\textcolor[rgb]{0.00,0.50,0.00}{##1}}}
\expandafter\def\csname PY@tok@nd\endcsname{\def\PY@tc##1{\textcolor[rgb]{0.67,0.13,1.00}{##1}}}
\expandafter\def\csname PY@tok@s\endcsname{\def\PY@tc##1{\textcolor[rgb]{0.73,0.13,0.13}{##1}}}
\expandafter\def\csname PY@tok@sd\endcsname{\let\PY@it=\textit\def\PY@tc##1{\textcolor[rgb]{0.73,0.13,0.13}{##1}}}
\expandafter\def\csname PY@tok@si\endcsname{\let\PY@bf=\textbf\def\PY@tc##1{\textcolor[rgb]{0.73,0.40,0.53}{##1}}}
\expandafter\def\csname PY@tok@se\endcsname{\let\PY@bf=\textbf\def\PY@tc##1{\textcolor[rgb]{0.73,0.40,0.13}{##1}}}
\expandafter\def\csname PY@tok@sr\endcsname{\def\PY@tc##1{\textcolor[rgb]{0.73,0.40,0.53}{##1}}}
\expandafter\def\csname PY@tok@ss\endcsname{\def\PY@tc##1{\textcolor[rgb]{0.10,0.09,0.49}{##1}}}
\expandafter\def\csname PY@tok@sx\endcsname{\def\PY@tc##1{\textcolor[rgb]{0.00,0.50,0.00}{##1}}}
\expandafter\def\csname PY@tok@m\endcsname{\def\PY@tc##1{\textcolor[rgb]{0.40,0.40,0.40}{##1}}}
\expandafter\def\csname PY@tok@gh\endcsname{\let\PY@bf=\textbf\def\PY@tc##1{\textcolor[rgb]{0.00,0.00,0.50}{##1}}}
\expandafter\def\csname PY@tok@gu\endcsname{\let\PY@bf=\textbf\def\PY@tc##1{\textcolor[rgb]{0.50,0.00,0.50}{##1}}}
\expandafter\def\csname PY@tok@gd\endcsname{\def\PY@tc##1{\textcolor[rgb]{0.63,0.00,0.00}{##1}}}
\expandafter\def\csname PY@tok@gi\endcsname{\def\PY@tc##1{\textcolor[rgb]{0.00,0.63,0.00}{##1}}}
\expandafter\def\csname PY@tok@gr\endcsname{\def\PY@tc##1{\textcolor[rgb]{1.00,0.00,0.00}{##1}}}
\expandafter\def\csname PY@tok@ge\endcsname{\let\PY@it=\textit}
\expandafter\def\csname PY@tok@gs\endcsname{\let\PY@bf=\textbf}
\expandafter\def\csname PY@tok@gp\endcsname{\let\PY@bf=\textbf\def\PY@tc##1{\textcolor[rgb]{0.00,0.00,0.50}{##1}}}
\expandafter\def\csname PY@tok@go\endcsname{\def\PY@tc##1{\textcolor[rgb]{0.53,0.53,0.53}{##1}}}
\expandafter\def\csname PY@tok@gt\endcsname{\def\PY@tc##1{\textcolor[rgb]{0.00,0.27,0.87}{##1}}}
\expandafter\def\csname PY@tok@err\endcsname{\def\PY@bc##1{\setlength{\fboxsep}{0pt}\fcolorbox[rgb]{1.00,0.00,0.00}{1,1,1}{\strut ##1}}}
\expandafter\def\csname PY@tok@kc\endcsname{\let\PY@bf=\textbf\def\PY@tc##1{\textcolor[rgb]{0.00,0.50,0.00}{##1}}}
\expandafter\def\csname PY@tok@kd\endcsname{\let\PY@bf=\textbf\def\PY@tc##1{\textcolor[rgb]{0.00,0.50,0.00}{##1}}}
\expandafter\def\csname PY@tok@kn\endcsname{\let\PY@bf=\textbf\def\PY@tc##1{\textcolor[rgb]{0.00,0.50,0.00}{##1}}}
\expandafter\def\csname PY@tok@kr\endcsname{\let\PY@bf=\textbf\def\PY@tc##1{\textcolor[rgb]{0.00,0.50,0.00}{##1}}}
\expandafter\def\csname PY@tok@bp\endcsname{\def\PY@tc##1{\textcolor[rgb]{0.00,0.50,0.00}{##1}}}
\expandafter\def\csname PY@tok@fm\endcsname{\def\PY@tc##1{\textcolor[rgb]{0.00,0.00,1.00}{##1}}}
\expandafter\def\csname PY@tok@vc\endcsname{\def\PY@tc##1{\textcolor[rgb]{0.10,0.09,0.49}{##1}}}
\expandafter\def\csname PY@tok@vg\endcsname{\def\PY@tc##1{\textcolor[rgb]{0.10,0.09,0.49}{##1}}}
\expandafter\def\csname PY@tok@vi\endcsname{\def\PY@tc##1{\textcolor[rgb]{0.10,0.09,0.49}{##1}}}
\expandafter\def\csname PY@tok@vm\endcsname{\def\PY@tc##1{\textcolor[rgb]{0.10,0.09,0.49}{##1}}}
\expandafter\def\csname PY@tok@sa\endcsname{\def\PY@tc##1{\textcolor[rgb]{0.73,0.13,0.13}{##1}}}
\expandafter\def\csname PY@tok@sb\endcsname{\def\PY@tc##1{\textcolor[rgb]{0.73,0.13,0.13}{##1}}}
\expandafter\def\csname PY@tok@sc\endcsname{\def\PY@tc##1{\textcolor[rgb]{0.73,0.13,0.13}{##1}}}
\expandafter\def\csname PY@tok@dl\endcsname{\def\PY@tc##1{\textcolor[rgb]{0.73,0.13,0.13}{##1}}}
\expandafter\def\csname PY@tok@s2\endcsname{\def\PY@tc##1{\textcolor[rgb]{0.73,0.13,0.13}{##1}}}
\expandafter\def\csname PY@tok@sh\endcsname{\def\PY@tc##1{\textcolor[rgb]{0.73,0.13,0.13}{##1}}}
\expandafter\def\csname PY@tok@s1\endcsname{\def\PY@tc##1{\textcolor[rgb]{0.73,0.13,0.13}{##1}}}
\expandafter\def\csname PY@tok@mb\endcsname{\def\PY@tc##1{\textcolor[rgb]{0.40,0.40,0.40}{##1}}}
\expandafter\def\csname PY@tok@mf\endcsname{\def\PY@tc##1{\textcolor[rgb]{0.40,0.40,0.40}{##1}}}
\expandafter\def\csname PY@tok@mh\endcsname{\def\PY@tc##1{\textcolor[rgb]{0.40,0.40,0.40}{##1}}}
\expandafter\def\csname PY@tok@mi\endcsname{\def\PY@tc##1{\textcolor[rgb]{0.40,0.40,0.40}{##1}}}
\expandafter\def\csname PY@tok@il\endcsname{\def\PY@tc##1{\textcolor[rgb]{0.40,0.40,0.40}{##1}}}
\expandafter\def\csname PY@tok@mo\endcsname{\def\PY@tc##1{\textcolor[rgb]{0.40,0.40,0.40}{##1}}}
\expandafter\def\csname PY@tok@ch\endcsname{\let\PY@it=\textit\def\PY@tc##1{\textcolor[rgb]{0.25,0.50,0.50}{##1}}}
\expandafter\def\csname PY@tok@cm\endcsname{\let\PY@it=\textit\def\PY@tc##1{\textcolor[rgb]{0.25,0.50,0.50}{##1}}}
\expandafter\def\csname PY@tok@cpf\endcsname{\let\PY@it=\textit\def\PY@tc##1{\textcolor[rgb]{0.25,0.50,0.50}{##1}}}
\expandafter\def\csname PY@tok@c1\endcsname{\let\PY@it=\textit\def\PY@tc##1{\textcolor[rgb]{0.25,0.50,0.50}{##1}}}
\expandafter\def\csname PY@tok@cs\endcsname{\let\PY@it=\textit\def\PY@tc##1{\textcolor[rgb]{0.25,0.50,0.50}{##1}}}

\def\PYZbs{\char`\\}
\def\PYZus{\char`\_}
\def\PYZob{\char`\{}
\def\PYZcb{\char`\}}
\def\PYZca{\char`\^}
\def\PYZam{\char`\&}
\def\PYZlt{\char`\<}
\def\PYZgt{\char`\>}
\def\PYZsh{\char`\#}
\def\PYZpc{\char`\%}
\def\PYZdl{\char`\$}
\def\PYZhy{\char`\-}
\def\PYZsq{\char`\'}
\def\PYZdq{\char`\"}
\def\PYZti{\char`\~}
% for compatibility with earlier versions
\def\PYZat{@}
\def\PYZlb{[}
\def\PYZrb{]}
\makeatother


    % Exact colors from NB
    \definecolor{incolor}{rgb}{0.0, 0.0, 0.5}
    \definecolor{outcolor}{rgb}{0.545, 0.0, 0.0}



    
    % Prevent overflowing lines due to hard-to-break entities
    \sloppy 
    % Setup hyperref package
    \hypersetup{
      breaklinks=true,  % so long urls are correctly broken across lines
      colorlinks=true,
      urlcolor=urlcolor,
      linkcolor=linkcolor,
      citecolor=citecolor,
      }
    % Slightly bigger margins than the latex defaults
    
    \geometry{verbose,tmargin=1in,bmargin=1in,lmargin=1in,rmargin=1in}
    
    

    \begin{document}
    
    
    \maketitle
    
    

    
    \section{Building a TensorFlow Container for Your Training and Hosting
Algorithms}\label{building-a-tensorflow-container-for-your-training-and-hosting-algorithms}

To use your own algorithms to train or deploy a model in Amazon
SageMaker, package them in Docker containers. By packaging an algorithm
in a container, you can use almost any code with Amazon SageMaker,
regardless of the programming language, environment, framework, or
dependencies. This example shows how to build a TensorFlow Docker
container and use it for training and inference in Amazon SageMaker.

\begin{enumerate}
\def\labelenumi{\arabic{enumi}.}
\tightlist
\item
  Section \ref{building-a-tensorflow-container-for-your_training-and-hosting-algorithms}
\item
  Section \ref{when-should-you-create-a-container-for-your-algorithm}
\item
  Section \ref{required-permissions}
\item
  Section \ref{the-example}
\item
  Section \ref{the-presentation}
\item
  Section \ref{part-1-packaging-and-uploading-your-algorithm-for-use-with-amazon-sagemaker}

  \begin{enumerate}
  \def\labelenumii{\arabic{enumii}.}
  \tightlist
  \item
    Section \ref{docker-overview}
  \item
    Section \ref{how-amazon-sagemaker-runs-your-docker-container}
  \item
    Section \ref{how-amazon-sagemaker-runs-your-container-during-training}

    \begin{enumerate}
    \def\labelenumiii{\arabic{enumiii}.}
    \tightlist
    \item
      Section \ref{the-input}
    \item
      Section \ref{the-output}
    \end{enumerate}
  \item
    Section \ref{how-amazon-sagemaker-runs-your-container-during-hosting}
  \item
    Section \ref{the-example-container}
  \item
    Section \ref{the-dockerfile}
  \item
    Section \ref{build-and-register-the-container}
  \end{enumerate}
\item
  Section \ref{test-your-algorithm-locally}
\item
  Section \ref{download-the-cifar-10-dataset}
\item
  Section \ref{train-locally-with-the-amazon-sagemaker-python-sdk}
\item
  Section \ref{fit-deploy-and-predict}
\item
  Section \ref{make-predictions-using-the-amazon-sagemaker-python-sdk}
\item
  Section \ref{part-2-training-and-hosting-your-algorithm-in-amazon-sagemaker}
\item
  Section \ref{set-up-the-environment}
\item
  Section \ref{create-the-session}
\item
  Section \ref{upload-the-data-for-training}
\item
  Section \ref{training-on--amazon-sagemaker}
\item
  Section \ref{optional-clean20up}\\
\item
  Section \ref{reference}
\end{enumerate}

\emph{or} I'm impatient, just Section \ref{the-dockerfile}!

\subsubsection{When Should You Create a Container for Your
Algorithm?}\label{when-should-you-create-a-container-for-your-algorithm}

It's not always necessary to create a container to use your own code in
Amazon SageMaker. If you use a framework that's supported by Amazon
SageMaker, such as Apache MXNet or TensorFlow, you can use the SDK entry
points for that framework to supply the Python code that implements your
algorithm. We regularly expand the set of supported frameworks, so
always check to see if the framework that your algorithm was written in
is supported.

However, even if there is SDK support for your framework, sometimes it's
more effective to build your own container. You might want to build your
own container if the code that implements your algorithm is complex or
if you need to make additions to the framework.

If your framework is supported, you might consider building your own
container for the following reasons:

\begin{itemize}
\tightlist
\item
  A specific version isn't supported.
\item
  You need to configure and install your dependencies and environment.
\item
  You use a different training or hosting solution than the one
  provided.
\end{itemize}

\subsubsection{Required Permissions}\label{required-permissions}

This notebook creates new repositories in Amazon Elastic Container
Registry (Amazon ECR), so you need permissions beyond those granted by
the \texttt{SageMakerFullAccess} permissions to run it. To add these
permissions, add the \texttt{AmazonEC2ContainerRegistryFullAccess}
managed policy to the AWS Identity and Access Management (IAM) role that
you used to start your notebook instance. The new permissions are
available immediately (you don't need to restart your notebook
instance).

\subsubsection{The Example}\label{the-example}

In this example, we show how to package a custom TensorFlow container
with an example written in Python. The example uses the CIFAR-10 dataset
for training and TensorFlow Serving for inference. You can use another
inference solution by modifying the Docker container.

We use a single image for both training and hosting because it's easier
to manage one image. If training and hosting have different
requirements, you might want to create a separate Dockerfile for each,
then build two images. Choose the approach that is easier to develop and
manage.

If you're using Amazon SageMaker only for training or only for hosting,
build only the required functionality into your container.

\subsubsection{The Presentation}\label{the-presentation}

This example is divided into two parts. The first explains how to
\emph{build} the container and the second explains how to \emph{use} the
container.

    \subsection{Part 1: Packaging and Uploading Your Algorithm for Use with
Amazon
SageMaker}\label{part-1-packaging-and-uploading-your-algorithm-for-use-with-amazon-sagemaker}

\subsubsection{Docker Overview}\label{docker-overview}

If you're familiar with Docker, you can skip to
Section \ref{how-amazon-sagemaker-runs-your-docker-container}
(\#How-Amazon-SageMaker-runs-your-Docker-container).

Docker has become very popular in programming and devops communities
because of its flexibility and its well-defined specification for how
code can be run in its containers. It is the underpinning of many
services built in the past few years, such as {[}Amazon Elastic
Container Service (Amazon ECS){]}. Although Docker containers are
unfamiliar to many data scientists, they aren't difficult to build and
use, and they can significantly simplify software package deployment.

You use Docker to package arbitrary code into an \emph{image} that is
totally self-contained. After creating the image, you use Docker to run
a \emph{container} based on that image. Running a container is just like
running a program, except that a container creates a fully
self-contained environment for the program to run in. Containers are
isolated from each other and from the host environment, so they run your
program the way it is set up, no matter where you run it.

Docker is more powerful than environment managers like Conda or
virtualenv because it is completely language independent, and because it
comprises your whole operating environment, including startup commands,
and environment variable.

A Docker container is like a virtual machine, but it is much lighter
weight. For example, a program running in a container can start in less
than a second, and many containers can run simultaneously on the same
physical or virtual machine instance.

Docker uses a simple file called a \texttt{Dockerfile} to specify how
the image is assembled. You'll see an example in this walkthrough. You
can build your Docker images based on Docker images that you've already
built or on images built by others.

Amazon SageMaker uses Docker to enable users to train and deploy
arbitrary algorithms. In Amazon SageMaker, Docker containers are invoked
one way for training and another, slightly different, way for hosting.

For more information about Docker, see the following:

\begin{itemize}
\tightlist
\item
  \href{http://www.docker.com}{Docker home page}
\item
  \href{https://docs.docker.com/get-started/}{Getting started with
  Docker}
\item
  \href{https://docs.docker.com/engine/reference/builder/}{Dockerfile
  reference}
\item
  \href{https://docs.docker.com/engine/reference/run/}{\texttt{docker\ run}
  reference}
\end{itemize}

\subsubsection{How Amazon SageMaker Runs Your Docker
Container}\label{how-amazon-sagemaker-runs-your-docker-container}

Because you can run the same image in training or hosting, Amazon
SageMaker runs your container with the argument \texttt{train} or
\texttt{serve}. How your container processes this argument depends on
the container.

\begin{itemize}
\tightlist
\item
  In this example, you don't define an \texttt{ENTRYPOINT} in the
  Dockerfile, so Docker runs the command
  \href{https://docs.aws.amazon.com/sagemaker/latest/dg/your-algorithms-training-algo.html}{\texttt{train}
  at training time} and
  \href{https://docs.aws.amazon.com/sagemaker/latest/dg/your-algorithms-inference-code.html}{\texttt{serve}
  at serving time}. In this example, you define these as executable
  Python scripts, but you could use any program that you want to start
  in that environment.
\item
  If you specify a program as an \texttt{ENTRYPOINT} in the Dockerfile,
  that program runs at startup, and its first argument is \texttt{train}
  or \texttt{serve}. The program looks at that argument and decides what
  to do.
\item
  If you are building separate containers for training and hosting (or
  building for only one or the other), you can define a program as an
  \texttt{ENTRYPOINT} in the Dockerfile and ignore (or verify) the first
  argument that is passed in.
\end{itemize}

\paragraph{How Amazon SageMaker Runs Your Container During
Training}\label{how-amazon-sagemaker-runs-your-container-during-training}

When Amazon SageMaker runs training, your \texttt{train} script runs as
in a regular Python program. The \texttt{/opt/ml} directory includes the
following files:

\begin{verbatim}
/opt/ml
├── input
│   ├── config
│   │   ├── hyperparameters.json
│   │   └── resourceConfig.json
│   └── data
│       └── <channel_name>
│           └── <input data>
├── model
│   └── <model files>
└── output
    └── failure
\end{verbatim}

\subparagraph{The Input}\label{the-input}

Input files provide the following: * \texttt{/opt/ml/input/config}
contains information to control how your program runs.
\texttt{hyperparameters.json} is a JSON-formatted dictionary that maps
hyperparameter names to values. These values are always strings, so you
might need to convert them. \texttt{resourceConfig.json} is a
JSON-formatted file that describes the network layout used for
distributed training. *
\texttt{/opt/ml/input/data/\textless{}channel\_name\textgreater{}/} (for
File mode) contains the input data for that channel. The channels are
created based on the call to the CreateTrainingJob operation, but it's
important that channels match algorithm expectations. The files for each
channel are copied from Amazon Simple Storage Service (Amazon S3) to
this directory, preserving the tree structure indicated by the S3 key
structure. *
\texttt{/opt/ml/input/data/\textless{}channel\_name\textgreater{}\_\textless{}epoch\_number\textgreater{}}
(for Pipe mode) is the pipe for a given epoch. Epochs start at zero and
increment by one each time you read them. There is no limit to the
number of epochs that you can run, but you must close each pipe before
reading the next epoch.

\subparagraph{The Output}\label{the-output}

There are two output directories: * \texttt{/opt/ml/model/} is the
directory where you write the model that your algorithm generates. Your
model can be in any format. It can be a single file or a whole directory
tree. Amazon SageMaker packages files in this directory into a
compressed tar archive file. This file is made available at the Amazon
S3 location returned in the \texttt{DescribeTrainingJob} response. *
\texttt{/opt/ml/output} is the directory where the algorithm can write a
file \texttt{failure} that describes why the job failed. The contents of
this file are returned in the \texttt{FailureReason} field of the
\texttt{DescribeTrainingJob} response. For jobs that succeed, there is
no reason to write this file because Amazon SageMaker ignores it.

\paragraph{How Amazon SageMaker Runs Your Container During
Hosting}\label{how-amazon-sagemaker-runs-your-container-during-hosting}

Hosting requires a very different model than training because hosting
reponds to inference requests that come in through HTTP. In this
example, we use \href{https://www.tensorflow.org/serving/}{TensorFlow
Serving}, but you can customize the hosting solution. For an example,
see
\href{https://github.com/awslabs/amazon-sagemaker-examples/blob/master/advanced_functionality/scikit_bring_your_own/scikit_bring_your_own.ipynb}{Python
serving stack within the scikit learn example}.

Amazon SageMaker hosting uses two URLs that are included in the
container:

\begin{itemize}
\tightlist
\item
  \texttt{/ping} receives \texttt{GET} requests from the infrastructure.
  If the container is accepting requests, your program returns 200.
\item
  \texttt{/invocations} is the endpoint that receives client inference
  \texttt{POST} requests. The format of the request and the response is
  up to the algorithm. If the client supplied \texttt{ContentType} and
  \texttt{Accept} headers, these are passed in too.
\end{itemize}

In the container, the model files are in the same place that they were
written to during training:

\begin{verbatim}
/opt/ml
└── model
    └── <model files>
\end{verbatim}

    \subsubsection{The Example Container}\label{the-example-container}

The \texttt{container} directory contains all of the components that you
need to package the example algorithm for Amazon SageMager:

\begin{verbatim}
.
├── Dockerfile
├── build_and_push.sh
└── cifar10
    ├── cifar10.py
    ├── resnet_model.py
    ├── nginx.conf
    ├── serve
    ├── train
\end{verbatim}

The components perform the following tasks:

\begin{itemize}
\tightlist
\item
  \textbf{\texttt{Dockerfile}} describes how to build your Docker
  container image.
\item
  \textbf{\texttt{build\_and\_push.sh}} is a script that uses the
  Dockerfile to build your container images and then pushes it to Amazon
  ECR. You invoke the commands directly later in this notebook, but you
  can copy and run the script for your own algorithms.
\item
  \textbf{\texttt{cifar10}} contains the files that are installed in the
  container.
\end{itemize}

For this simple application, you install only five files in the
container:

\begin{itemize}
\tightlist
\item
  \textbf{\texttt{cifar10.py}} is the program that implements your
  training algorithm.
\item
  \textbf{\texttt{resnet\_model.py}} is the program that contains your
  ResNet model.
\item
  \textbf{\texttt{nginx.conf}} is the configuration file for the nginx
  front end. Generally, you should be able to take this file as is.
\item
  \textbf{\texttt{serve}} is the program that is started when the
  container is started for hosting. It launches nginx and loads your
  exported model with TensorFlow Serving.
\item
  \textbf{\texttt{train}} is the program that is invoked when the
  container is run for training. Our implementation of this script
  invokes cifar10.py with hyperparameter values retrieved from
  /opt/ml/input/config/hyperparameters.json. We do this to avoid having
  to modify the training algorithm program.
\end{itemize}

You might need only five files, but if you have many supporting
routines, you might want to install more.

This is the standard structure of our Python containers, although you
are free to choose a different toolset and, therefore, could have a
different layout. If you're writing in a different programming language,
your layout will depend on the framework and tools that you choose.

You probably will want to change two files for your application:
\texttt{train} and \texttt{serve}.

    \subsubsection{The Dockerfile}\label{the-dockerfile}

The Dockerfile describes the image that you want to build. It describes
the complete operating system installation of the system that you want
to run. A running Docker container is much lighter than a full operating
system, because it uses Linux on the host machine for basic operations.

For the Python science stack, start with an official TensorFlow Docker
image and run the standard tools to install TensorFlow Serving. Then add
the code that implements your specific algorithm to the container, and
set up the right environment for it to run under.

Here's the Dockerfile for this example:

    \begin{Verbatim}[commandchars=\\\{\}]
{\color{incolor}In [{\color{incolor}1}]:} \PY{o}{!}cat container/Dockerfile
\end{Verbatim}


    \begin{Verbatim}[commandchars=\\\{\}]
\# Copyright 2017-2018 Amazon.com, Inc. or its affiliates. All Rights Reserved.

\#

\# Licensed under the Apache License, Version 2.0 (the "License"). You

\# may not use this file except in compliance with the License. A copy of

\# the License is located at

\#

\#     http://aws.amazon.com/apache2.0/

\#

\# or in the "license" file accompanying this file. This file is

\# distributed on an "AS IS" BASIS, WITHOUT WARRANTIES OR CONDITIONS OF

\# ANY KIND, either express or implied. See the License for the specific

\# language governing permissions and limitations under the License.



\# For more information on creating a Dockerfile

\# https://docs.docker.com/compose/gettingstarted/\#step-2-create-a-dockerfile

FROM tensorflow/tensorflow:1.8.0-py3



RUN apt-get update \&\& apt-get install -y --no-install-recommends nginx curl



\# Download TensorFlow Serving

\# https://www.tensorflow.org/serving/setup\#installing\_the\_modelserver

RUN echo "deb [arch=amd64] http://storage.googleapis.com/tensorflow-serving-apt stable tensorflow-model-server tensorflow-model-server-universal" | tee /etc/apt/sources.list.d/tensorflow-serving.list

RUN curl https://storage.googleapis.com/tensorflow-serving-apt/tensorflow-serving.release.pub.gpg | apt-key add -

RUN apt-get update \&\& apt-get install tensorflow-model-server



ENV PATH="/opt/ml/code:\$\{PATH\}"



\# /opt/ml and all subdirectories are utilized by SageMaker, we use the /code subdirectory to store our user code.

COPY /cifar10 /opt/ml/code

WORKDIR /opt/ml/code
    \end{Verbatim}

    \subsubsection{Build and Register the
Container}\label{build-and-register-the-container}

The following shell code uses \texttt{docker\ build} to build the
container image and \texttt{docker\ push} to push the container image to
Amazon ECR. This code is also available as the shell script
\texttt{container/build-and-push.sh}, which you can run as
\texttt{build-and-push.sh\ tensorflow-cifar10-example} to build the
image \texttt{tensorflow-cifar10-example}.

The code looks for an Amazon ECR repository in the account that you're
using and the current default AWS Region (if you're using an Amazon
SageMaker notebook instance, this is the Region where the notebook
instance was created). If there is no Amazon ECR repository, the script
creates it.

    \begin{Verbatim}[commandchars=\\\{\}]
{\color{incolor}In [{\color{incolor} }]:} \PYZpc{}\PYZpc{}sh
        
        \PY{c+c1}{\PYZsh{} The name of our algorithm}
        \PY{n+nv}{algorithm\PYZus{}name}\PY{o}{=}tensorflow\PYZhy{}cifar10\PYZhy{}example
        
        \PY{n+nb}{cd} container
        
        chmod +x cifar10/train
        chmod +x cifar10/serve
        
        \PY{n+nv}{account}\PY{o}{=}\PY{k}{\PYZdl{}(}aws sts get\PYZhy{}caller\PYZhy{}identity \PYZhy{}\PYZhy{}query Account \PYZhy{}\PYZhy{}output text\PY{k}{)}
        
        \PY{c+c1}{\PYZsh{} Get the region defined in the current configuration (default to us\PYZhy{}west\PYZhy{}2 if none defined)}
        \PY{n+nv}{region}\PY{o}{=}\PY{k}{\PYZdl{}(}aws configure get region\PY{k}{)}
        \PY{n+nv}{region}\PY{o}{=}\PY{l+s+si}{\PYZdl{}\PYZob{}}\PY{n+nv}{region}\PY{k}{:\PYZhy{}}\PY{n+nv}{us}\PY{p}{\PYZhy{}west\PYZhy{}2}\PY{l+s+si}{\PYZcb{}}
        
        \PY{n+nv}{fullname}\PY{o}{=}\PY{l+s+s2}{\PYZdq{}}\PY{l+s+si}{\PYZdl{}\PYZob{}}\PY{n+nv}{account}\PY{l+s+si}{\PYZcb{}}\PY{l+s+s2}{.dkr.ecr.}\PY{l+s+si}{\PYZdl{}\PYZob{}}\PY{n+nv}{region}\PY{l+s+si}{\PYZcb{}}\PY{l+s+s2}{.amazonaws.com/}\PY{l+s+si}{\PYZdl{}\PYZob{}}\PY{n+nv}{algorithm\PYZus{}name}\PY{l+s+si}{\PYZcb{}}\PY{l+s+s2}{:latest}\PY{l+s+s2}{\PYZdq{}}
        
        \PY{c+c1}{\PYZsh{} If the repository doesn\PYZsq{}t exist in ECR, create it.}
        
        aws ecr describe\PYZhy{}repositories \PYZhy{}\PYZhy{}repository\PYZhy{}names \PY{l+s+s2}{\PYZdq{}}\PY{l+s+si}{\PYZdl{}\PYZob{}}\PY{n+nv}{algorithm\PYZus{}name}\PY{l+s+si}{\PYZcb{}}\PY{l+s+s2}{\PYZdq{}} \PYZgt{} /dev/null \PY{l+m}{2}\PYZgt{}\PY{p}{\PYZam{}}\PY{l+m}{1}
        
        \PY{k}{if} \PY{o}{[} \PY{n+nv}{\PYZdl{}?} \PYZhy{}ne \PY{l+m}{0} \PY{o}{]}
        \PY{k}{then}
            aws ecr create\PYZhy{}repository \PYZhy{}\PYZhy{}repository\PYZhy{}name \PY{l+s+s2}{\PYZdq{}}\PY{l+s+si}{\PYZdl{}\PYZob{}}\PY{n+nv}{algorithm\PYZus{}name}\PY{l+s+si}{\PYZcb{}}\PY{l+s+s2}{\PYZdq{}} \PYZgt{} /dev/null
        \PY{k}{fi}
        
        \PY{c+c1}{\PYZsh{} Get the login command from ECR and execute it directly}
        \PY{k}{\PYZdl{}(}aws ecr get\PYZhy{}login \PYZhy{}\PYZhy{}region \PY{l+s+si}{\PYZdl{}\PYZob{}}\PY{n+nv}{region}\PY{l+s+si}{\PYZcb{}} \PYZhy{}\PYZhy{}no\PYZhy{}include\PYZhy{}email\PY{k}{)}
        
        \PY{c+c1}{\PYZsh{} Build the docker image locally with the image name and then push it to ECR}
        \PY{c+c1}{\PYZsh{} with the full name.}
        
        docker build  \PYZhy{}t \PY{l+s+si}{\PYZdl{}\PYZob{}}\PY{n+nv}{algorithm\PYZus{}name}\PY{l+s+si}{\PYZcb{}} .
        docker tag \PY{l+s+si}{\PYZdl{}\PYZob{}}\PY{n+nv}{algorithm\PYZus{}name}\PY{l+s+si}{\PYZcb{}} \PY{l+s+si}{\PYZdl{}\PYZob{}}\PY{n+nv}{fullname}\PY{l+s+si}{\PYZcb{}}
        
        docker push \PY{l+s+si}{\PYZdl{}\PYZob{}}\PY{n+nv}{fullname}\PY{l+s+si}{\PYZcb{}}
\end{Verbatim}


    \subsubsection{Test the Algorithm
Locally}\label{test-the-algorithm-locally}

When you're packaging your first algorithm for use with Amazon
SageMaker, it's a good idea to test it to make sure it's working
correctly. You use the
\href{https://github.com/aws/sagemaker-python-sdk}{Amazon SageMaker
Python SDK} to test both locally and on Amazon SageMaker. To test our
algorithm, you need to download our dataset.

For more examples of using the Amazon SageMaker Python SDK, see
\href{https://github.com/awslabs/amazon-sagemaker-examples/tree/master/sagemaker-python-sdk}{Amazon
SageMaker Examples}.

    \subsubsection{Download the CIFAR-10
Dataset}\label{download-the-cifar-10-dataset}

The training algorithm expects training data to be in
\href{https://www.tensorflow.org/guide/datasets}{TFRecords} file format.
TFRecords format is a simple record-oriented binary format that many
TensorFlow applications use for training data. The following Python
script downloads the CIFAR-10 dataset and converts them into TFRecords.
It is adapted from the
\href{https://github.com/tensorflow/models/tree/master/tutorials/image/cifar10_estimator}{official
TensorFlow CIFAR-10 example}.

    \begin{Verbatim}[commandchars=\\\{\}]
{\color{incolor}In [{\color{incolor} }]:} \PY{o}{!} python utils/generate\PYZus{}cifar10\PYZus{}tfrecords.py \PYZhy{}\PYZhy{}data\PYZhy{}dir\PY{o}{=}/tmp/cifar\PYZhy{}10\PYZhy{}data
\end{Verbatim}


    \begin{Verbatim}[commandchars=\\\{\}]
{\color{incolor}In [{\color{incolor}2}]:} \PY{c+c1}{\PYZsh{} There should be three tfrecords. (eval, train, validation)}
        \PY{o}{!} ls /tmp/cifar\PYZhy{}10\PYZhy{}data
\end{Verbatim}


    \begin{Verbatim}[commandchars=\\\{\}]
eval.tfrecords	train.tfrecords  validation.tfrecords


    \end{Verbatim}

    \subsubsection{Train Locally with the Amazon SageMaker Python
SDK}\label{train-locally-with-the-amazon-sagemaker-python-sdk}

To represent training, you use the Estimator class. You need to
configure the following: 1. IAM role - The AWS execution role. 2.
train\_instance\_count - The number of instances to use for training. 3.
train\_instance\_type - The type of instance to use for training. For
training locally, specify \texttt{local}. 4. image\_name - The name of
the custom TensorFlow Docker image that we created. 5. hyperparameters -
The hyperparameters that we want to pass.

To set up the IAM role, you use a helper function in the Amazon
SageMaker Python SDK. The function gets metadata from the notebook
instance, so if you run it outside of a notebook instance, it throws an
exception. To run the function outside of a notebook instance, you must
provide an IAM role that has the required permissions. For more
information, see Section \ref{permissions}.

    \begin{Verbatim}[commandchars=\\\{\}]
{\color{incolor}In [{\color{incolor}3}]:} \PY{k+kn}{from} \PY{n+nn}{sagemaker} \PY{k}{import} \PY{n}{get\PYZus{}execution\PYZus{}role}
        
        \PY{n}{role} \PY{o}{=} \PY{n}{get\PYZus{}execution\PYZus{}role}\PY{p}{(}\PY{p}{)}
\end{Verbatim}


    \subsubsection{Fit, Deploy, and Predict}\label{fit-deploy-and-predict}

Now you can call \texttt{fit()} with the path to your local copy of the
CIFAR-10 dataset prefixed with \texttt{file://}. This invokes the
TensorFlow container with \texttt{train} and passes in your
hyperparameters and other metadata as .json files in
/opt/ml/input/config within the container.

After training succeeds, the training algorithm outputs a trained model
to the /opt/ml/model directory, which is used to handle predictions.

You can then call \texttt{deploy()} with the instance\_count and
instance\_type, \texttt{1} and \texttt{local}, respectively. This
invokes the TensorFlow container with \texttt{serve}, which sets up your
container to handle prediction requests through TensorFlow Serving. A
predictor is returned, which you use to make inferences against your
trained model.

After you get your prediction, you can delete the endpoint.

We recommend testing and training your training algorithm locally first,
because it iterates faster and is easier to debug.

    \begin{Verbatim}[commandchars=\\\{\}]
{\color{incolor}In [{\color{incolor}4}]:} \PY{c+c1}{\PYZsh{} Lets set up our SageMaker notebook instance for local mode.}
        \PY{o}{!}/bin/bash ./utils/setup.sh
\end{Verbatim}


    \begin{Verbatim}[commandchars=\\\{\}]
SageMaker instance route table setup is ok. We are good to go.

SageMaker instance routing for Docker is ok. We are good to go!


    \end{Verbatim}

    \begin{Verbatim}[commandchars=\\\{\}]
{\color{incolor}In [{\color{incolor} }]:} \PY{k+kn}{from} \PY{n+nn}{sagemaker}\PY{n+nn}{.}\PY{n+nn}{estimator} \PY{k}{import} \PY{n}{Estimator}
        
        \PY{n}{hyperparameters} \PY{o}{=} \PY{p}{\PYZob{}}\PY{l+s+s1}{\PYZsq{}}\PY{l+s+s1}{train\PYZhy{}steps}\PY{l+s+s1}{\PYZsq{}}\PY{p}{:} \PY{l+m+mi}{100}\PY{p}{\PYZcb{}}
        
        \PY{n}{instance\PYZus{}type} \PY{o}{=} \PY{l+s+s1}{\PYZsq{}}\PY{l+s+s1}{local}\PY{l+s+s1}{\PYZsq{}}
        
        \PY{n}{estimator} \PY{o}{=} \PY{n}{Estimator}\PY{p}{(}\PY{n}{role}\PY{o}{=}\PY{n}{role}\PY{p}{,}
                              \PY{n}{train\PYZus{}instance\PYZus{}count}\PY{o}{=}\PY{l+m+mi}{1}\PY{p}{,}
                              \PY{n}{train\PYZus{}instance\PYZus{}type}\PY{o}{=}\PY{n}{instance\PYZus{}type}\PY{p}{,}
                              \PY{n}{image\PYZus{}name}\PY{o}{=}\PY{l+s+s1}{\PYZsq{}}\PY{l+s+s1}{tensorflow\PYZus{}cifar10\PYZus{}example:latest}\PY{l+s+s1}{\PYZsq{}}\PY{p}{,}
                              \PY{n}{hyperparameters}\PY{o}{=}\PY{n}{hyperparameters}\PY{p}{)}
        
        \PY{n}{estimator}\PY{o}{.}\PY{n}{fit}\PY{p}{(}\PY{l+s+s1}{\PYZsq{}}\PY{l+s+s1}{file:///tmp/cifar\PYZhy{}10\PYZhy{}data}\PY{l+s+s1}{\PYZsq{}}\PY{p}{)}
        
        \PY{n}{predictor} \PY{o}{=} \PY{n}{estimator}\PY{o}{.}\PY{n}{deploy}\PY{p}{(}\PY{l+m+mi}{1}\PY{p}{,} \PY{n}{instance\PYZus{}type}\PY{p}{)}
\end{Verbatim}


    \subsubsection{Make Predictions Using the Amazon SageMaker Python
SDK}\label{make-predictions-using-the-amazon-sagemaker-python-sdk}

To make predictions, you use an image that is converted into JSON format
with OpenCV. You send this as an inference request. You also install
OpenCV to deserialize the image that is used to make predictions.

The JSON response is the probability that the image belongs to each of
the 10 classes and the most likely class that it belongs to. For a list
of classes, see the
\href{https://www.cs.toronto.edu/~kriz/cifar.html}{CIFAR-10 website}. We
didn't train the model for long, so we aren't expecting very accurate
results.

    \begin{Verbatim}[commandchars=\\\{\}]
{\color{incolor}In [{\color{incolor} }]:} \PY{o}{!} pip install opencv\PYZhy{}python
\end{Verbatim}


    \begin{Verbatim}[commandchars=\\\{\}]
{\color{incolor}In [{\color{incolor}6}]:} \PY{k+kn}{import} \PY{n+nn}{cv2}
        \PY{k+kn}{import} \PY{n+nn}{numpy}
        
        \PY{k+kn}{from} \PY{n+nn}{sagemaker}\PY{n+nn}{.}\PY{n+nn}{predictor} \PY{k}{import} \PY{n}{json\PYZus{}serializer}\PY{p}{,} \PY{n}{json\PYZus{}deserializer}
        
        \PY{n}{image} \PY{o}{=} \PY{n}{cv2}\PY{o}{.}\PY{n}{imread}\PY{p}{(}\PY{l+s+s2}{\PYZdq{}}\PY{l+s+s2}{data/cat.png}\PY{l+s+s2}{\PYZdq{}}\PY{p}{,} \PY{l+m+mi}{1}\PY{p}{)}
        
        \PY{c+c1}{\PYZsh{} resize, as our model is expecting images in 32x32.}
        \PY{n}{image} \PY{o}{=} \PY{n}{cv2}\PY{o}{.}\PY{n}{resize}\PY{p}{(}\PY{n}{image}\PY{p}{,} \PY{p}{(}\PY{l+m+mi}{32}\PY{p}{,} \PY{l+m+mi}{32}\PY{p}{)}\PY{p}{)}
        
        \PY{n}{data} \PY{o}{=} \PY{p}{\PYZob{}}\PY{l+s+s1}{\PYZsq{}}\PY{l+s+s1}{instances}\PY{l+s+s1}{\PYZsq{}}\PY{p}{:} \PY{n}{numpy}\PY{o}{.}\PY{n}{asarray}\PY{p}{(}\PY{n}{image}\PY{p}{)}\PY{o}{.}\PY{n}{astype}\PY{p}{(}\PY{n+nb}{float}\PY{p}{)}\PY{o}{.}\PY{n}{tolist}\PY{p}{(}\PY{p}{)}\PY{p}{\PYZcb{}}
        
        \PY{c+c1}{\PYZsh{} The request and response format is JSON for TensorFlow Serving.}
        \PY{c+c1}{\PYZsh{} For more information: https://www.tensorflow.org/serving/api\PYZus{}rest\PYZsh{}predict\PYZus{}api}
        \PY{n}{predictor}\PY{o}{.}\PY{n}{accept} \PY{o}{=} \PY{l+s+s1}{\PYZsq{}}\PY{l+s+s1}{application/json}\PY{l+s+s1}{\PYZsq{}}
        \PY{n}{predictor}\PY{o}{.}\PY{n}{content\PYZus{}type} \PY{o}{=} \PY{l+s+s1}{\PYZsq{}}\PY{l+s+s1}{application/json}\PY{l+s+s1}{\PYZsq{}}
        
        \PY{n}{predictor}\PY{o}{.}\PY{n}{serializer} \PY{o}{=} \PY{n}{json\PYZus{}serializer}
        \PY{n}{predictor}\PY{o}{.}\PY{n}{deserializer} \PY{o}{=} \PY{n}{json\PYZus{}deserializer}
        
        \PY{c+c1}{\PYZsh{} For more information on the predictor class.}
        \PY{c+c1}{\PYZsh{} https://github.com/aws/sagemaker\PYZhy{}python\PYZhy{}sdk/blob/master/src/sagemaker/predictor.py}
        \PY{n}{predictor}\PY{o}{.}\PY{n}{predict}\PY{p}{(}\PY{n}{data}\PY{p}{)}
\end{Verbatim}


    \begin{Verbatim}[commandchars=\\\{\}]
\textcolor{ansi-cyan}{algo-1-L58J2\_1  |} 172.18.0.1 - - [03/Aug/2018:22:32:52 +0000] "POST /invocations HTTP/1.1" 200 229 "-" "-"


    \end{Verbatim}

\begin{Verbatim}[commandchars=\\\{\}]
{\color{outcolor}Out[{\color{outcolor}6}]:} \{'predictions': [\{'probabilities': [2.29861e-05,
            0.0104983,
            0.147974,
            0.01538,
            0.0478089,
            0.00164997,
            0.758483,
            0.0164191,
            0.00125304,
            0.000510801],
           'classes': 6\}]\}
\end{Verbatim}
            
    \begin{Verbatim}[commandchars=\\\{\}]
{\color{incolor}In [{\color{incolor} }]:} \PY{n}{predictor}\PY{o}{.}\PY{n}{delete\PYZus{}endpoint}\PY{p}{(}\PY{p}{)}
\end{Verbatim}


    \subsection{Part 2: Training and Hosting Your Algorithm in Amazon
SageMaker}\label{part-2-training-and-hosting-your-algorithm-in-amazon-sagemaker}

After packaging your container, you can use it to train and serve
models.

\subsubsection{Set Up the Environment}\label{set-up-the-environment}

Specify the S3 bucket to use and the IAM role needed to work with Amazon
SageMaker.

    \begin{Verbatim}[commandchars=\\\{\}]
{\color{incolor}In [{\color{incolor} }]:} \PY{c+c1}{\PYZsh{} S3 prefix}
        \PY{n}{prefix} \PY{o}{=} \PY{l+s+s1}{\PYZsq{}}\PY{l+s+s1}{DEMO\PYZhy{}tensorflow\PYZhy{}cifar10}\PY{l+s+s1}{\PYZsq{}}
\end{Verbatim}


    \subsubsection{Create the Session}\label{create-the-session}

The session remembers connection parameters to Amazon SageMaker. you
perform all of our Amazon SageMaker operations in the session.

    \begin{Verbatim}[commandchars=\\\{\}]
{\color{incolor}In [{\color{incolor} }]:} \PY{k+kn}{import} \PY{n+nn}{sagemaker} \PY{k}{as} \PY{n+nn}{sage}
        
        \PY{n}{sess} \PY{o}{=} \PY{n}{sage}\PY{o}{.}\PY{n}{Session}\PY{p}{(}\PY{p}{)}
\end{Verbatim}


    \subsubsection{Upload the Data for
Training}\label{upload-the-data-for-training}

To upload the data to a default S3 bucket, you use the tools provided by
the Amazon SageMaker Python SDK.

    \begin{Verbatim}[commandchars=\\\{\}]
{\color{incolor}In [{\color{incolor} }]:} \PY{n}{WORK\PYZus{}DIRECTORY} \PY{o}{=} \PY{l+s+s1}{\PYZsq{}}\PY{l+s+s1}{/tmp/cifar\PYZhy{}10\PYZhy{}data}\PY{l+s+s1}{\PYZsq{}}
        
        \PY{n}{data\PYZus{}location} \PY{o}{=} \PY{n}{sess}\PY{o}{.}\PY{n}{upload\PYZus{}data}\PY{p}{(}\PY{n}{WORK\PYZus{}DIRECTORY}\PY{p}{,} \PY{n}{key\PYZus{}prefix}\PY{o}{=}\PY{n}{prefix}\PY{p}{)}
\end{Verbatim}


    \subsubsection{Train on Amazon
SageMaker}\label{train-on-amazon-sagemaker}

To train a model on Amazon SageMaker, you use the Amazon SageMaker
Python SDK similar to the way that you used it to train a model locally.
You do the following:

\begin{enumerate}
\def\labelenumi{\arabic{enumi}.}
\tightlist
\item
  Change the train\_instance\_type from \texttt{local} to one of the
  \href{https://aws.amazon.com/sagemaker/pricing/instance-types/}{supported
  EC2 instance types}.
\item
  Specify the URL of the Amazon ECR image that you pushed.
\item
  Make sure that your local training dataset is in Amazon S3 and that
  the S3 URL to the dataset is passed into the \texttt{fit()} call.
\end{enumerate}

Begin by fetching the URL of the Amazon ECR image.

    \begin{Verbatim}[commandchars=\\\{\}]
{\color{incolor}In [{\color{incolor} }]:} \PY{k+kn}{import} \PY{n+nn}{boto3}
        
        \PY{n}{client} \PY{o}{=} \PY{n}{boto3}\PY{o}{.}\PY{n}{client}\PY{p}{(}\PY{l+s+s1}{\PYZsq{}}\PY{l+s+s1}{sts}\PY{l+s+s1}{\PYZsq{}}\PY{p}{)}
        \PY{n}{account} \PY{o}{=} \PY{n}{client}\PY{o}{.}\PY{n}{get\PYZus{}caller\PYZus{}identity}\PY{p}{(}\PY{p}{)}\PY{p}{[}\PY{l+s+s1}{\PYZsq{}}\PY{l+s+s1}{Account}\PY{l+s+s1}{\PYZsq{}}\PY{p}{]}
        
        \PY{n}{my\PYZus{}session} \PY{o}{=} \PY{n}{boto3}\PY{o}{.}\PY{n}{session}\PY{o}{.}\PY{n}{Session}\PY{p}{(}\PY{p}{)}
        \PY{n}{region} \PY{o}{=} \PY{n}{my\PYZus{}session}\PY{o}{.}\PY{n}{region\PYZus{}name}
        
        \PY{n}{algorithm\PYZus{}name} \PY{o}{=} \PY{l+s+s1}{\PYZsq{}}\PY{l+s+s1}{tensorflow\PYZhy{}cifar10\PYZhy{}example}\PY{l+s+s1}{\PYZsq{}}
        
        \PY{n}{ecr\PYZus{}image} \PY{o}{=} \PY{l+s+s1}{\PYZsq{}}\PY{l+s+si}{\PYZob{}\PYZcb{}}\PY{l+s+s1}{.dkr.ecr.}\PY{l+s+si}{\PYZob{}\PYZcb{}}\PY{l+s+s1}{.amazonaws.com/}\PY{l+s+si}{\PYZob{}\PYZcb{}}\PY{l+s+s1}{:latest}\PY{l+s+s1}{\PYZsq{}}\PY{o}{.}\PY{n}{format}\PY{p}{(}\PY{n}{account}\PY{p}{,} \PY{n}{region}\PY{p}{,} \PY{n}{algorithm\PYZus{}name}\PY{p}{)}
        
        \PY{n+nb}{print}\PY{p}{(}\PY{n}{ecr\PYZus{}image}\PY{p}{)}
\end{Verbatim}


    \begin{Verbatim}[commandchars=\\\{\}]
{\color{incolor}In [{\color{incolor} }]:} \PY{k+kn}{from} \PY{n+nn}{sagemaker}\PY{n+nn}{.}\PY{n+nn}{estimator} \PY{k}{import} \PY{n}{Estimator}
        
        \PY{n}{hyperparameters} \PY{o}{=} \PY{p}{\PYZob{}}\PY{l+s+s1}{\PYZsq{}}\PY{l+s+s1}{train\PYZhy{}steps}\PY{l+s+s1}{\PYZsq{}}\PY{p}{:} \PY{l+m+mi}{100}\PY{p}{\PYZcb{}}
        
        \PY{n}{instance\PYZus{}type} \PY{o}{=} \PY{l+s+s1}{\PYZsq{}}\PY{l+s+s1}{ml.m4.xlarge}\PY{l+s+s1}{\PYZsq{}}
        
        \PY{n}{estimator} \PY{o}{=} \PY{n}{Estimator}\PY{p}{(}\PY{n}{role}\PY{o}{=}\PY{n}{role}\PY{p}{,}
                              \PY{n}{train\PYZus{}instance\PYZus{}count}\PY{o}{=}\PY{l+m+mi}{1}\PY{p}{,}
                              \PY{n}{train\PYZus{}instance\PYZus{}type}\PY{o}{=}\PY{n}{instance\PYZus{}type}\PY{p}{,}
                              \PY{n}{image\PYZus{}name}\PY{o}{=}\PY{n}{ecr\PYZus{}image}\PY{p}{,}
                              \PY{n}{hyperparameters}\PY{o}{=}\PY{n}{hyperparameters}\PY{p}{)}
        
        \PY{n}{estimator}\PY{o}{.}\PY{n}{fit}\PY{p}{(}\PY{n}{data\PYZus{}location}\PY{p}{)}
        
        \PY{n}{predictor} \PY{o}{=} \PY{n}{estimator}\PY{o}{.}\PY{n}{deploy}\PY{p}{(}\PY{l+m+mi}{1}\PY{p}{,} \PY{n}{instance\PYZus{}type}\PY{p}{)}
\end{Verbatim}


    \begin{Verbatim}[commandchars=\\\{\}]
{\color{incolor}In [{\color{incolor} }]:} \PY{n}{image} \PY{o}{=} \PY{n}{cv2}\PY{o}{.}\PY{n}{imread}\PY{p}{(}\PY{l+s+s2}{\PYZdq{}}\PY{l+s+s2}{data/cat.png}\PY{l+s+s2}{\PYZdq{}}\PY{p}{,} \PY{l+m+mi}{1}\PY{p}{)}
        
        \PY{c+c1}{\PYZsh{} resize, as our model is expecting images in 32x32.}
        \PY{n}{image} \PY{o}{=} \PY{n}{cv2}\PY{o}{.}\PY{n}{resize}\PY{p}{(}\PY{n}{image}\PY{p}{,} \PY{p}{(}\PY{l+m+mi}{32}\PY{p}{,} \PY{l+m+mi}{32}\PY{p}{)}\PY{p}{)}
        
        \PY{n}{data} \PY{o}{=} \PY{p}{\PYZob{}}\PY{l+s+s1}{\PYZsq{}}\PY{l+s+s1}{instances}\PY{l+s+s1}{\PYZsq{}}\PY{p}{:} \PY{n}{numpy}\PY{o}{.}\PY{n}{asarray}\PY{p}{(}\PY{n}{image}\PY{p}{)}\PY{o}{.}\PY{n}{astype}\PY{p}{(}\PY{n+nb}{float}\PY{p}{)}\PY{o}{.}\PY{n}{tolist}\PY{p}{(}\PY{p}{)}\PY{p}{\PYZcb{}}
        
        \PY{n}{predictor}\PY{o}{.}\PY{n}{accept} \PY{o}{=} \PY{l+s+s1}{\PYZsq{}}\PY{l+s+s1}{application/json}\PY{l+s+s1}{\PYZsq{}}
        \PY{n}{predictor}\PY{o}{.}\PY{n}{content\PYZus{}type} \PY{o}{=} \PY{l+s+s1}{\PYZsq{}}\PY{l+s+s1}{application/json}\PY{l+s+s1}{\PYZsq{}}
        
        \PY{n}{predictor}\PY{o}{.}\PY{n}{serializer} \PY{o}{=} \PY{n}{json\PYZus{}serializer}
        \PY{n}{predictor}\PY{o}{.}\PY{n}{deserializer} \PY{o}{=} \PY{n}{json\PYZus{}deserializer}
        
        \PY{n}{predictor}\PY{o}{.}\PY{n}{predict}\PY{p}{(}\PY{n}{data}\PY{p}{)}
\end{Verbatim}


    \subsubsection{Optional: Clean Up}\label{optional-clean-up}

You can see all of the training jobs, models, and endpoints that you
created in the Amazon SageMaker console in your AWS account. When you're
done with the endpoint, delete it to avoid accruing unnecessary charges.

    \begin{Verbatim}[commandchars=\\\{\}]
{\color{incolor}In [{\color{incolor} }]:} \PY{n}{predictor}\PY{o}{.}\PY{n}{delete\PYZus{}endpoint}\PY{p}{(}\PY{p}{)}
\end{Verbatim}


    \subsection{Reference}\label{reference}

\begin{itemize}
\tightlist
\item
  \href{https://docs.aws.amazon.com/sagemaker/latest/dg/your-algorithms-training-algo.html}{How
  Amazon SageMaker Runs Your Training Image}
\item
  \href{https://docs.aws.amazon.com/sagemaker/latest/dg/your-algorithms-inference-code.html}{How
  Amazon SageMaker Runs Your Inference Image}
\item
  \href{https://www.cs.toronto.edu/~kriz/cifar.html}{CIFAR-10 Dataset}
\item
  \href{https://github.com/aws/sagemaker-python-sdk}{Amazon SageMaker
  Python SDK}
\item
  \href{https://docs.docker.com/engine/reference/builder/}{Dockerfile}
\item
  \href{https://github.com/awslabs/amazon-sagemaker-examples/blob/master/advanced_functionality/scikit_bring_your_own/scikit_bring_your_own.ipynb}{scikit-bring-your-own}
\end{itemize}


    % Add a bibliography block to the postdoc
    
    
    
    \end{document}
